\documentclass[../main.tex]{subfile}

\addbibresource{../bibs/lsc-variational.bib}

\begin{document}
\chapter{Lower---Semicontinuity and Material Science}

\emph{Contributed by Samuel Wallace\footnote{contact: \href{mailto:swalla24@uic.edu}{swalla24@uic.edu}}} \\

Materials are most precisely characterized by their atomic arrangements.
If you had perfectly accurate data on the positions and arrangements of atoms at any one particular time and if you had a very powerful computer, you could go ahead and calculate absolutely everything about an object of that material: electrical and magnetic properties, mechanical properties, and thermodynamic quantities.
However, such a powerful computer and that data is far out of reach currently.
Therefore to figure out properties of materials, we need a more feasible model. \\

A well---studied model for materials is a continuum model.
In this model, we assume a material is made of continuous ``stuff'' that entirely fills the space the object takes up, and this ``stuff'' has a more--or-less continuously distributed properties.
Mathematically, the object is seen as an open set $\Omega \subset \mathbb{R}^n$, and various functions are defined on this set to represent the various properties. \\

In this section, I'll be analyzing some mechanical properties of elastic objects through the continuum modeling lens.

\section{Mathematical Modeling of Elastostatics and the Direct Method}

Elastic materials are those that deform when forces are applied, and return to their initial state after the forces are removed.
A model for elastic materials is the \emph{hyper---elastic} model: given a map $y: \Omega \to \mathbb{R}^3$ that represents the position and orientation of our material sample in space, we have a function $W: \mathrm{Mat}_{n\times n} \to \mathbb{R}$ so that $W(\nabla y(x))$ is the energy density of the deformation at a point $x$.
We may then integrate this function over $\Omega$ to get the total energy of our deformation. \\

A seemingly simple problem to analyze is how elastic objects deform when they are not moving; these are \emph{elastostatic} problems.
The key modeling assumption is that these are global or local minimizers of the energy.
The idea behind this assumption is that derivative of energy is forces, and if there were forces on this object it would move; so the derivative of energy must be zero. \\

This explanation is heuristic, but can be justified, to various degrees, mathematically.
For now, we will leave it as an assumption, but keep the result: our hope is to understand the quantity

\begin{equation}
  \label{eq:min-energy}
  \min_{y: \Omega \to \mathbb{R}^n} \int_{\Omega}W( \nabla y) dx
\end{equation}

under some assumptions on $y$; for example, we could impose continuity or regularity assumptions, or boundary conditions, or applied forces, and so on.
For our discussion now, we won't worry about which kind of restrictions we put on $y$, but we will make more abstract assumptions on the restrictions in relation to the outcome of the problem \eqref{eq:min-energy}.
For example, we could ask that our restrictions give us a unique solution. \\

How could we verify that an elastostatics problem has a minimizer?




\printbibliography

\end{document}
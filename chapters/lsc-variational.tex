\documentclass[../main.tex]{subfile}

\addbibresource{../bibs/lsc-variational.bib}

\begin{document}
\chapter{Lower---Semicontinuity and Material Science}

\emph{Contributed by Samuel Wallace\footnote{contact: \href{mailto:swalla24@uic.edu}{swalla24@uic.edu}}} \\

Materials are most precisely characterized by their atomic arrangements.
If you had perfectly accurate data on the positions and arrangements of atoms at any one particular time and if you had a very powerful computer, you could go ahead and calculate absolutely everything about an object of that material: electrical and magnetic properties, mechanical properties, and thermodynamic quantities.
However, such a powerful computer and that data is far out of reach currently.
Therefore to figure out properties of materials, we need a more feasible model. \\

A well---studied model for materials is a continuum model.
In this model, we assume a material is made of continuous ``stuff'' that entirely fills the space the object takes up, and this ``stuff'' has a more--or-less continuously distributed properties.
Mathematically, the object is seen as an open set $\Omega \subset \mathbb{R}^n$, and various functions are defined on this set to represent the various properties. \\

In this section, I'll be analyzing some mechanical properties of elastic objects through the continuum modeling lens.

\section{Mathematical Modeling of Elastostatics and the Direct Method}

Elastic materials are those that deform when forces are applied, and return to their initial state after the forces are removed.
A model for elastic materials is the \emph{hyper---elastic} model: given a map $y: \Omega \to \mathbb{R}^3$ that represents the position and orientation of our material sample in space, we have a function $W: \mathrm{Mat}_{n\times n} \to \mathbb{R}$ so that $W(\nabla y(x))$ is the energy density of the deformation at a point $x$.
We may then integrate this function over $\Omega$ to get the total energy of our deformation. \\

A seemingly simple problem to analyze is how elastic objects deform when they are not moving; these are \emph{elastostatic} problems.
The key modeling assumption is that these are global or local minimizers of the energy.
The idea behind this assumption is that derivative of energy is forces, and if there were forces on this object it would move; so the derivative of energy must be zero. \\

This explanation is heuristic, but can be justified, to various degrees, mathematically.
For now, we will leave it as an assumption, but keep the result: our hope is to understand the quantity

\begin{equation}
  \label{eq:min-energy}
  \min_{y: \Omega \to \mathbb{R}^n} E(y) = \min_{y: \Omega \to \mathbb{R}^n} \int_{\Omega}W( \nabla y) dx
\end{equation}

under some assumptions on $y$; for example, we could impose continuity or regularity assumptions, or boundary conditions, or applied forces, and so on.
For our discussion now, we won't worry about which kind of restrictions we put on $y$, but we will make more abstract assumptions on the restrictions in relation to the outcome of the problem \eqref{eq:min-energy}.
For example, we could ask that our restrictions give us a unique solution. \\

How could we verify that an elastostatics problem has a minimizer? To do so, we'll introduce the Direct Method in the Calculus of Variations.
This is a proof technique for showing existence of a global (i.e. over all allowed functions) minimizer  of a variational problem. \\

The basic strategy to to the direct method is to take any sequence of functions whose energy approaches the minimum, show it converges in some relevant topology, and show the limit actually gets the minimum value.
The convergence is usually gotten by choosing a coarser topology on the relevant functions, and the last step requires, at least, lower--semicontinuity of the integrand; for us, $W$.
To see an example of its application, let's take a look at the classic direct method proof from \cite{evans_partial_2010}.
Before that, we need a small, but significant result:

\begin{lem}
  If $W$ is a convex function on the set of matrices, then it is lower semi--continuous on $W^{1,p}(\Omega)$ in its weak topology for $1 < p < \infty$, where $W^{1,p}$ is a Sobolev space (cf. \ref{apdx:sobolev}).
\end{lem}

\begin{proof}
  Let $y_k \harpoonarrowup y$ in $W^{1,p}(\Omega)$; we wish to show that $\liminf_{k \to \infty} E(y_k) \geq E(y)$.
  Then it follows from Soboelv inequalities and Rellich--Kondrachov that $y_k \harpoonarrowup y$ strongly in $L^p$, and their gradients converge weakly in $L^p$. 
  Because $W$ is convex, we have the pointwise inequality $W(\nabla y_k) - W(\nabla y) \geq  DW(\nabla y) \cdot (\nabla y_{k} - \nabla y)$, so that integrating,
  \begin{equation*}
    \int_{\Omega} W(\nabla y_k) \geq \int_{\Omega} W(\nabla y) dx + \int_{\Omega} DW ( \nabla y) \cdot (\nabla y_k - \nabla y) dx
  \end{equation*}

  And the last term goes to zero as $\nabla y_{k} \harpoonarrowup \nabla y$ in $L^p$.
\end{proof}

\begin{thm}
  Assume the following on $W$:
  \begin{itemize}
  \item $W(F) \geq \alpha \abs{F}^q - \beta$ for some constant $\alpha, \beta > 0$ and $1 < p < \infty$
  \item $W$ is a convex function,
  \item The set of admissible functions are non--empty.
  \end{itemize}
\end{thm}



\printbibliography

\end{document}